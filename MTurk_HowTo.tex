\listfiles
\documentclass[10pt]{article}
\usepackage{geometry}               
\geometry{letterpaper}                
\usepackage[parfill]{parskip}    
\usepackage{graphicx}
\usepackage{amssymb}
\usepackage{epstopdf}
\usepackage{setspace}
\usepackage{cite}
\usepackage{natbib}
\usepackage{xcolor}
\usepackage{hyperref}
\hypersetup{
citebordercolor=green
}
\onehalfspacing
\DeclareGraphicsRule{.tif}{png}{.png}{`convert #1 `dirname #1`/`basename #1 .tif`.png}
\title{Guide to the MTurk Command Line Tools}
\author{Harvard Department of Government}
\date{Last updated: March 11, 2012}                                          
\begin{document}
\maketitle

\begin{quotation}
The MTurk Command Line Tools are a set of CLIs that expand on the functions available through the MTurk website. This guide prepares the user to use the Command Line Tools and explains how to use two additional functions, NotifyCSV and PayBonusCSV. If you have any questions, contact clucas@fas.harvard.edu.
\end{quotation}

\section*{Getting Started}

\begin{enumerate}
	\item You will need an Amazon Web Services (AWS) account and an Amazon Mechanical Turk Requester account in order to use the Command Line Tools. 
	\item If you do not have an AWS account, click this \href{https://aws-portal.amazon.com/gp/aws/developer/registration/index.html}{link.}
	\item If you do not have an MTurk Requester account, click this \href{https://www.amazon.com/gp/aws/ssop/index.html?awscbctx=&awscbid=1PAVRHWQ0W17WVS868G2&awscredential=&awsnoclientpipeline=true&awsstrict=false&awsturknosubway=true&wa=wsignin1.0&wctx=&wreply=https\%3A\%2F\%2Frequester.mturk.com\%2Fend_signin&wtrealm=1PAVRHWQ0W17WVS868G2&awssig=8EwI\%2FStQzSbYuJTv3SrgxdaEqAA\%3D}{link.}
\end{enumerate}

\section*{Downloading the Command Line Tools}

\begin{enumerate}
	\item The Command Line Tools use the Java Runtime Environment (JRE). The JRE is probably installed on your computer already. However, you can check by visiting this 					\href{http://javatester.org/version.html}{website} or by calling \verb+java -version+ at the command line. Though the Command Line Tools were written for JRE 5 (1.5.X) and are not 100\% compatible with JRE 6 (1.6.X), NotifyCSV and PayBonusCSV function properly with JRE 6. 
	\item If your computer does not have the JRE, you will need to install it. If you use Windows, you can download the JRE with the Command Line Tools. This is not an option for users with a *nix machine. If you need to download the JRE or prefer to install it apart from the Command Line Tools, click this \href{java.sun.com/javase/downloads/index_jdk5.jsp}{link} for JRE 5 or this \href{http://java.com/en/download/index.jsp}{link} for JRE 6.
	\item Download the Command Line Tools:
	\begin{itemize}
		\item Windows users: You have three download options. 
			\begin{itemize}
				\item Command Line Tools with Setup, includes the Java Runtime Environment (JRE)
				\item Command Line Tools without Setup, includes JRE 
				\item Command Line Tools without Setup, no JRE
			\end{itemize}
		\item *nix users: You have one option, which does not include a setup file or the JRE
	\end{itemize}
	\item Download the appropriate version \href{http://aws.amazon.com/developertools/694}{here.}
\end{enumerate}

\section*{Installing the Command Line Tools}

\begin{quotation}
If you downloaded and installed \verb+mech-turk-setup.exe+ (the Windows download with the setup and the JRE), you can skip this section.
\end{quotation}

\begin{enumerate}
	\item Uncompress the zipped file into a convenient folder. 
	\item Open the file $\sim$\verb+\aws-mturk-clt-1.3.0\bin\mturk.properties+ in a text editor.
	\item Set your access key and your secret key to reflect your AWS identifiers.
	\begin{itemize}
		\item You can find your access key and secret key \href{https://aws-portal.amazon.com/gp/aws/securityCredentials}{here.}
		\item Set the keys according to the following example. 
		\begin{verbatim}
		access_key = Your AWS Access Key
		secret_key = Your Secret Key
		\end{verbatim}
	\end{itemize}
	\item Save and close the file.
	\item *nix users: You must set the \verb+MTURK_CMD_HOME+ environment variable and the \verb+JAVA_HOME+ environment variable. 
		\begin{itemize}
			\item At the command line, type and enter \verb+export MTURK_CMD_HOME=/where/you/installed/the/CLT+ to set the \verb+MTURK_CMD_HOME+ environment variable. 
			\item You must set the \verb+JAVA_HOME+ environment variable to point to your Java Standard Edition Runtime Environment (JRE) installation location. If you do not know your JRE installation location, try \verb+which java+. If that doesn't work, try 
\verb+locate java+. 
			\item To set the \verb+JAVA_HOME+ environment variable, type and enter \verb+export JAVA_HOME=/usr/local/jdk-1.5+ at the command line. 
		\end{itemize}
\end{enumerate}

\section*{Using the Command Line Tools in the Developer Sandbox}

\begin{quotation}
The Developer Sandbox is a simulated environment that allows developers to test their Amazon Mechanical Turk solutions for free. The Command Line Tools are set by default to work with the production site. However, the can also be used in the Developer Sandbox. 
\end{quotation}

\begin{enumerate}
	\item \item Open the file $\sim$\verb+\aws-mturk-clt-1.3.0\bin\mturk.properties+ in a text editor.
	\item Comment out the Production site \verb+service_url+ with a "\#."
	\item Uncomment the Developer Sandbox \verb+service_url+ by removing the "\#."
\end{enumerate}

\section*{Running NotifyCSV}

\begin{enumerate}
	\item Before running the script, you must set up the csv input file such that the first line contains a comma-separated list consisting of the message subject, the message body, and the worker ID. Subsequent lines must contain only the next worker ID to receive the same message. 
	\item Messages can only be sent to a worker ID that you have interacted with in the past.
	\item From the command line, navigate to the directory containing the \verb+build.xml+. 
	\item Type and enter \verb+ant+ at the command line. The output should say ``BUILD SUCCESSFUL'' near the end. If not, install Apache Ant \href{http://ant.apache.org/}{here.}
	\item Type and enter \verb+ant notifycsv -Dnoitifycsv=/location/of/the/csv+.
\end{enumerate}

\section*{Running PayBonusCSV}

\begin{enumerate}
	\item Before running the script, you must set up the csv input file such that each line contains a comma-separated list consisting of the assignment ID, the worker ID, the bonus amount and the reason for paying (in that order). 
	\item From the command line, navigate to the directory containing the \verb+build.xml+. 
	\item Type and enter \verb+ant+ at the command line. The output should say ``BUILD SUCCESSFUL'' near the end. If not, install Apache Ant \href{http://ant.apache.org/}{here.}
	\item Type and enter \verb+ant ant paybonuscsv -Dbonuscsv=/location/of/the/csv+.
\end{enumerate}


\end{document}








